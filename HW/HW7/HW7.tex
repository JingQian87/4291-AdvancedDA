\documentclass[]{article}
\usepackage{lmodern}
\usepackage{amssymb,amsmath}
\usepackage{ifxetex,ifluatex}
\usepackage{fixltx2e} % provides \textsubscript
\ifnum 0\ifxetex 1\fi\ifluatex 1\fi=0 % if pdftex
  \usepackage[T1]{fontenc}
  \usepackage[utf8]{inputenc}
\else % if luatex or xelatex
  \ifxetex
    \usepackage{mathspec}
  \else
    \usepackage{fontspec}
  \fi
  \defaultfontfeatures{Ligatures=TeX,Scale=MatchLowercase}
\fi
% use upquote if available, for straight quotes in verbatim environments
\IfFileExists{upquote.sty}{\usepackage{upquote}}{}
% use microtype if available
\IfFileExists{microtype.sty}{%
\usepackage{microtype}
\UseMicrotypeSet[protrusion]{basicmath} % disable protrusion for tt fonts
}{}
\usepackage[margin=1in]{geometry}
\usepackage{hyperref}
\hypersetup{unicode=true,
            pdftitle={HW7-ADA},
            pdfborder={0 0 0},
            breaklinks=true}
\urlstyle{same}  % don't use monospace font for urls
\usepackage{color}
\usepackage{fancyvrb}
\newcommand{\VerbBar}{|}
\newcommand{\VERB}{\Verb[commandchars=\\\{\}]}
\DefineVerbatimEnvironment{Highlighting}{Verbatim}{commandchars=\\\{\}}
% Add ',fontsize=\small' for more characters per line
\usepackage{framed}
\definecolor{shadecolor}{RGB}{248,248,248}
\newenvironment{Shaded}{\begin{snugshade}}{\end{snugshade}}
\newcommand{\AlertTok}[1]{\textcolor[rgb]{0.94,0.16,0.16}{#1}}
\newcommand{\AnnotationTok}[1]{\textcolor[rgb]{0.56,0.35,0.01}{\textbf{\textit{#1}}}}
\newcommand{\AttributeTok}[1]{\textcolor[rgb]{0.77,0.63,0.00}{#1}}
\newcommand{\BaseNTok}[1]{\textcolor[rgb]{0.00,0.00,0.81}{#1}}
\newcommand{\BuiltInTok}[1]{#1}
\newcommand{\CharTok}[1]{\textcolor[rgb]{0.31,0.60,0.02}{#1}}
\newcommand{\CommentTok}[1]{\textcolor[rgb]{0.56,0.35,0.01}{\textit{#1}}}
\newcommand{\CommentVarTok}[1]{\textcolor[rgb]{0.56,0.35,0.01}{\textbf{\textit{#1}}}}
\newcommand{\ConstantTok}[1]{\textcolor[rgb]{0.00,0.00,0.00}{#1}}
\newcommand{\ControlFlowTok}[1]{\textcolor[rgb]{0.13,0.29,0.53}{\textbf{#1}}}
\newcommand{\DataTypeTok}[1]{\textcolor[rgb]{0.13,0.29,0.53}{#1}}
\newcommand{\DecValTok}[1]{\textcolor[rgb]{0.00,0.00,0.81}{#1}}
\newcommand{\DocumentationTok}[1]{\textcolor[rgb]{0.56,0.35,0.01}{\textbf{\textit{#1}}}}
\newcommand{\ErrorTok}[1]{\textcolor[rgb]{0.64,0.00,0.00}{\textbf{#1}}}
\newcommand{\ExtensionTok}[1]{#1}
\newcommand{\FloatTok}[1]{\textcolor[rgb]{0.00,0.00,0.81}{#1}}
\newcommand{\FunctionTok}[1]{\textcolor[rgb]{0.00,0.00,0.00}{#1}}
\newcommand{\ImportTok}[1]{#1}
\newcommand{\InformationTok}[1]{\textcolor[rgb]{0.56,0.35,0.01}{\textbf{\textit{#1}}}}
\newcommand{\KeywordTok}[1]{\textcolor[rgb]{0.13,0.29,0.53}{\textbf{#1}}}
\newcommand{\NormalTok}[1]{#1}
\newcommand{\OperatorTok}[1]{\textcolor[rgb]{0.81,0.36,0.00}{\textbf{#1}}}
\newcommand{\OtherTok}[1]{\textcolor[rgb]{0.56,0.35,0.01}{#1}}
\newcommand{\PreprocessorTok}[1]{\textcolor[rgb]{0.56,0.35,0.01}{\textit{#1}}}
\newcommand{\RegionMarkerTok}[1]{#1}
\newcommand{\SpecialCharTok}[1]{\textcolor[rgb]{0.00,0.00,0.00}{#1}}
\newcommand{\SpecialStringTok}[1]{\textcolor[rgb]{0.31,0.60,0.02}{#1}}
\newcommand{\StringTok}[1]{\textcolor[rgb]{0.31,0.60,0.02}{#1}}
\newcommand{\VariableTok}[1]{\textcolor[rgb]{0.00,0.00,0.00}{#1}}
\newcommand{\VerbatimStringTok}[1]{\textcolor[rgb]{0.31,0.60,0.02}{#1}}
\newcommand{\WarningTok}[1]{\textcolor[rgb]{0.56,0.35,0.01}{\textbf{\textit{#1}}}}
\usepackage{graphicx,grffile}
\makeatletter
\def\maxwidth{\ifdim\Gin@nat@width>\linewidth\linewidth\else\Gin@nat@width\fi}
\def\maxheight{\ifdim\Gin@nat@height>\textheight\textheight\else\Gin@nat@height\fi}
\makeatother
% Scale images if necessary, so that they will not overflow the page
% margins by default, and it is still possible to overwrite the defaults
% using explicit options in \includegraphics[width, height, ...]{}
\setkeys{Gin}{width=\maxwidth,height=\maxheight,keepaspectratio}
\IfFileExists{parskip.sty}{%
\usepackage{parskip}
}{% else
\setlength{\parindent}{0pt}
\setlength{\parskip}{6pt plus 2pt minus 1pt}
}
\setlength{\emergencystretch}{3em}  % prevent overfull lines
\providecommand{\tightlist}{%
  \setlength{\itemsep}{0pt}\setlength{\parskip}{0pt}}
\setcounter{secnumdepth}{0}
% Redefines (sub)paragraphs to behave more like sections
\ifx\paragraph\undefined\else
\let\oldparagraph\paragraph
\renewcommand{\paragraph}[1]{\oldparagraph{#1}\mbox{}}
\fi
\ifx\subparagraph\undefined\else
\let\oldsubparagraph\subparagraph
\renewcommand{\subparagraph}[1]{\oldsubparagraph{#1}\mbox{}}
\fi

%%% Use protect on footnotes to avoid problems with footnotes in titles
\let\rmarkdownfootnote\footnote%
\def\footnote{\protect\rmarkdownfootnote}

%%% Change title format to be more compact
\usepackage{titling}

% Create subtitle command for use in maketitle
\newcommand{\subtitle}[1]{
  \posttitle{
    \begin{center}\large#1\end{center}
    }
}

\setlength{\droptitle}{-2em}

  \title{HW7-ADA}
    \pretitle{\vspace{\droptitle}\centering\huge}
  \posttitle{\par}
    \author{}
    \preauthor{}\postauthor{}
    \date{}
    \predate{}\postdate{}
  

\begin{document}
\maketitle

\hypertarget{jing-qian-jq2282}{%
\section{Jing Qian (jq2282)}\label{jing-qian-jq2282}}

\hypertarget{a.-estimate-the-probability-that-a-male-protestant-and-republican-supports-laws-legalizing-abortion.-estimate-the-probability-that-female-catholic-and-democrat-supports-laws-legalizing-abortion.}{%
\paragraph{1.a. Estimate the probability that a male, protestant and
republican supports laws legalizing abortion. Estimate the probability
that female, catholic and democrat supports laws legalizing
abortion.}\label{a.-estimate-the-probability-that-a-male-protestant-and-republican-supports-laws-legalizing-abortion.-estimate-the-probability-that-female-catholic-and-democrat-supports-laws-legalizing-abortion.}}

The estimated logit model is logit(\(\hat{\pi}\)) = 0.11 + 0.16G −
0.57\(R_1\) − 0.66\(R_2\) + 0.47\(P_1\) − 1.67\(P_2\). Then the
description that a male, protestant and republican supports laws
legalizing abortion corresponds to logit(\(\hat{\pi_1}\)) = 0.11 +
0.16(0) − 0.57(1) − 0.66(0) + 0.47(0) − 1.67(1) = -2.13. So the
probability that a male, protestant and republican supports laws
legalizing abortion is
\(\hat{\pi_1} = \frac{e^{-2.13}}{1+e^{-2.13}} = 0.106\).

The description that a female, catholic and democrat supports laws
legalizing abortion corresponds to logit(\(\hat{\pi_2}\)) = 0.11 +
0.16(1) − 0.57(0) − 0.66(1) + 0.47(1) − 1.67(0) = 0.08. So the
probability that a male, protestant and republican supports laws
legalizing abortion is
\(\hat{\pi_2} = \frac{e^{0.08}}{1+e^{0.08}} = 0.520\).

\hypertarget{b.-interpret-b_1-0.16-and-b_2-0.57.}{%
\paragraph{\texorpdfstring{1.b. Interpret \(b_1\) = 0.16 and \(b_2\) =
−0.57.}{1.b. Interpret b\_1 = 0.16 and b\_2 = −0.57.}}\label{b.-interpret-b_1-0.16-and-b_2-0.57.}}

Interpretation of \(b_1\) = 0.16: If holding religion affiliation and
political party fixed, the odds that a female supports laws legalizing
abortion is estimated to be \(e^{0.16} = 1.174\) times the odds that a
male does.

Interpretation of \(b_2 = -0.57\): If holding gender and political party
fixed, the odds that a protestant supports laws legalizing abortion is
estimated to be \(e^{-0.57} = 0.566\) times the odds that a
non-protestant does.

\hypertarget{c.-if-seb_1-0.064-construct-a-95-confidence-interval-for-b_1-and-interpret-your-result.}{%
\paragraph{\texorpdfstring{1.c. If SE(\(b_1\)) = 0.064 construct a 95\%
confidence interval for \(b_1\) and interpret your
result.}{1.c. If SE(b\_1) = 0.064 construct a 95\% confidence interval for b\_1 and interpret your result.}}\label{c.-if-seb_1-0.064-construct-a-95-confidence-interval-for-b_1-and-interpret-your-result.}}

The 95\% confidence interval for \(b_1\) is
\(b_1 \pm 1.96\ \rm{SE}(b_1) = 0.16 \pm 1.96*0.064 = [0.035, 0.285]\).

So the 95\% confidence interval for \(e^{b_1}\) is \([1.036, 1.330]\).
We are 95\% confident that if holding religion affiliation and political
party fixed, the ratio between odds that a female supports laws
legalizing abortion and odds that a male does is a number between 1.036
and 1.330.

Since the 95\% confidence interval for \(b_1\) does not cover 0, we may
infer that gender is significant to the individual's attitude towards
laws legalizing abortion.

\hypertarget{d.-test-h_0-beta_1-0-against-h_a-beta_1-neq-0.}{%
\paragraph{\texorpdfstring{1.d. Test \(H_0\): \(\beta_1\) = 0 against
\(H_a\):
\(\beta_1 \neq 0\).}{1.d. Test H\_0: \textbackslash{}beta\_1 = 0 against H\_a: \textbackslash{}beta\_1 \textbackslash{}neq 0.}}\label{d.-test-h_0-beta_1-0-against-h_a-beta_1-neq-0.}}

Here we use \(Z = \frac{b_1 - 0}{\rm{SE}(b_1)}\) to test hypothesis.
\(Z = \frac{0.16}{0.064} = 2.5\), which is larger than
\(Z_{\alpha/2} = 1.96\). So we could reject \(H_0: \beta_1 = 0\) with
\(\alpha = 0.05\), which means that gender is a significant variable to
whether an individual supports laws legalizing abortion.

\hypertarget{e.-if-seb_2-0.38-construct-a-95-confidence-interval-for-b_2-and-interpret-your-result.}{%
\paragraph{\texorpdfstring{1.e. If SE(\(b_2\)) = 0.38 construct a 95\%
confidence interval for \(b_2\) and interpret your
result.}{1.e. If SE(b\_2) = 0.38 construct a 95\% confidence interval for b\_2 and interpret your result.}}\label{e.-if-seb_2-0.38-construct-a-95-confidence-interval-for-b_2-and-interpret-your-result.}}

The 95\% confidence interval for \(b_2\) is
\(b_2 \pm 1.96\ \rm{SE}(b_2) = -0.57 \pm 1.96*0.38 = [-1.315, 0.175]\).

So the 95\% confidence interval for \(e^{b_2}\) is \([0.268, 1.191]\).
We are 95\% confident that if holding gender and political party fixed,
the ratio between odds that a protestant supports laws legalizing
abortion and odds that a non-protestant does is a number between 0.268
and 1.191.

Since the 95\% confidence interval for \(b_2\) covers 0, we may infer
that whether an individual is protestant or not is not significant to
his or her attitude towards laws legalizing abortion.

\hypertarget{f.-test-h_0-beta_2-0-against-h_a-beta_2-neq-0.}{%
\paragraph{\texorpdfstring{1.f. Test \(H_0\): \(\beta_2\) = 0 against
\(H_a\):
\(\beta_2 \neq 0\).}{1.f. Test H\_0: \textbackslash{}beta\_2 = 0 against H\_a: \textbackslash{}beta\_2 \textbackslash{}neq 0.}}\label{f.-test-h_0-beta_2-0-against-h_a-beta_2-neq-0.}}

Here we use \(Z = \frac{b_2 - 0}{\rm{SE}(b_2)}\) to test hypothesis.
\(Z = \frac{-0.57}{0.38} = -1.5\), whose absolute value is smaller than
\(Z_{\alpha/2} = 1.96\). So we could not reject \(H_0: \beta_2 = 0\)
with \(\alpha = 0.05\), which means that whether an individual is a
protestant or not is not significant to his or her attitude towards laws
legalizing abortion.

\hypertarget{a.-estimate-beta_1-and-beta_2-and-interpret-your-result.}{%
\paragraph{\texorpdfstring{2.a. Estimate \(\beta_1\) and \(\beta_2\) and
interpret your
result.}{2.a. Estimate \textbackslash{}beta\_1 and \textbackslash{}beta\_2 and interpret your result.}}\label{a.-estimate-beta_1-and-beta_2-and-interpret-your-result.}}

\begin{Shaded}
\begin{Highlighting}[]
\NormalTok{data<-}\KeywordTok{read.csv}\NormalTok{(}\StringTok{"~/Desktop/AdvancedDA/HW/HW7/adolescent.csv"}\NormalTok{,}\DataTypeTok{header=}\OtherTok{TRUE}\NormalTok{,}\DataTypeTok{sep=}\StringTok{','}\NormalTok{)}
\end{Highlighting}
\end{Shaded}

\begin{verbatim}
## Warning in read.table(file = file, header = header, sep = sep, quote =
## quote, : incomplete final line found by readTableHeader on '~/Desktop/
## AdvancedDA/HW/HW7/adolescent.csv'
\end{verbatim}

\begin{Shaded}
\begin{Highlighting}[]
\KeywordTok{attach}\NormalTok{(data)}
\end{Highlighting}
\end{Shaded}

\begin{Shaded}
\begin{Highlighting}[]
\NormalTok{logit1<-}\KeywordTok{glm}\NormalTok{(}\KeywordTok{cbind}\NormalTok{(Yes,No)}\OperatorTok{~}\KeywordTok{factor}\NormalTok{(Gender)}\OperatorTok{+}\KeywordTok{factor}\NormalTok{(Race),}\DataTypeTok{family=}\NormalTok{binomial)}
\NormalTok{logit1}
\end{Highlighting}
\end{Shaded}

\begin{verbatim}
## 
## Call:  glm(formula = cbind(Yes, No) ~ factor(Gender) + factor(Race), 
##     family = binomial)
## 
## Coefficients:
##        (Intercept)  factor(Gender)Male   factor(Race)White  
##            -0.4555              0.6478             -1.3135  
## 
## Degrees of Freedom: 3 Total (i.e. Null);  1 Residual
## Null Deviance:       37.52 
## Residual Deviance: 0.05835   AIC: 25.19
\end{verbatim}

Here Gender = 1 if gender is male and Gender = 0 if gender is female.
Race = 1 if race is white and Race = 0 if race is black. Then we get
estimated coefficients: \(b_1 = 0.6478,\ b_2 = -1.3135\).

\(b_1 = 0.6478\) means that if holding race fixed, the odds that a male
adolescent has had intercourse is estimated to be \(e^{0.6478} = 1.911\)
times the odds that a female does.

\(b_2 = -1.3135\) means that if holding gender fixed, the odds that a
white adolescent has had intercourse is estimated to be
\(e^{-1.3135} = 0.269\) times the odds that a black does.

\hypertarget{b.-construct-a-95-confidence-interval-to-describe-the-effect-of-gender-on-the-odds-of-intercourse-controlling-for-race-i.e.-construct-a-95-interval-for-ebeta_1.-interpret-your-result.}{%
\paragraph{\texorpdfstring{2.b. Construct a 95\% confidence interval to
describe the effect of gender on the odds of Intercourse controlling for
race (i.e.~construct a 95\% interval for \(e^{\beta_1}\)). Interpret
your
result.}{2.b. Construct a 95\% confidence interval to describe the effect of gender on the odds of Intercourse controlling for race (i.e.~construct a 95\% interval for e\^{}\{\textbackslash{}beta\_1\}). Interpret your result.}}\label{b.-construct-a-95-confidence-interval-to-describe-the-effect-of-gender-on-the-odds-of-intercourse-controlling-for-race-i.e.-construct-a-95-interval-for-ebeta_1.-interpret-your-result.}}

\begin{Shaded}
\begin{Highlighting}[]
\KeywordTok{summary}\NormalTok{(logit1)}
\end{Highlighting}
\end{Shaded}

\begin{verbatim}
## 
## Call:
## glm(formula = cbind(Yes, No) ~ factor(Gender) + factor(Race), 
##     family = binomial)
## 
## Deviance Residuals: 
##        1         2         3         4  
## -0.08867   0.10840   0.14143  -0.13687  
## 
## Coefficients:
##                    Estimate Std. Error z value Pr(>|z|)    
## (Intercept)         -0.4555     0.2221  -2.050  0.04032 *  
## factor(Gender)Male   0.6478     0.2250   2.879  0.00399 ** 
## factor(Race)White   -1.3135     0.2378  -5.524 3.32e-08 ***
## ---
## Signif. codes:  0 '***' 0.001 '**' 0.01 '*' 0.05 '.' 0.1 ' ' 1
## 
## (Dispersion parameter for binomial family taken to be 1)
## 
##     Null deviance: 37.516984  on 3  degrees of freedom
## Residual deviance:  0.058349  on 1  degrees of freedom
## AIC: 25.186
## 
## Number of Fisher Scoring iterations: 3
\end{verbatim}

From the result above, SE(\(b_1\)) = 0.2250 for \(b_1 = 0.6478\). The
95\% confidence interval for \(b_1\) is:
\(b_1 \pm 1.96\ \rm{SE}(b_1) = 0.6478 \pm 1.96*0.2250 = [0.207, 1.089]\).

So the 95\% interval for \(e^{b_1}\) is \([1.230, 2.971]\). We are 95\%
confident that if holding race fixed, the ratio between odds that a male
adolescent has had intercourse and odds that a female does is a number
between 1.230 and 2.971.

Since the 95\% confidence interval for \(e^{b_1}\) does not cover 1, we
may infer that gender is significant to having sexual intercourse.

\hypertarget{c.-test-h_0-beta_1-0-against-h_a-beta_1-neq-0.-use-alpha-0.05.}{%
\paragraph{\texorpdfstring{2.c. Test \(H_0\): \(\beta_1\) = 0 against
\(H_a\): \(\beta_1 \neq 0\). Use \(\alpha\) =
0.05.}{2.c. Test H\_0: \textbackslash{}beta\_1 = 0 against H\_a: \textbackslash{}beta\_1 \textbackslash{}neq 0. Use \textbackslash{}alpha = 0.05.}}\label{c.-test-h_0-beta_1-0-against-h_a-beta_1-neq-0.-use-alpha-0.05.}}

Here we use \(Z = \frac{b_1 - 0}{\rm{SE}(b_1)}\) to test hypothesis.
\(Z = \frac{0.6478}{0.2250} = 2.879\), which is larger than
\(Z_{\alpha/2} = 1.96\). So we could reject \(H_0: \beta_1 = 0\) with
\(\alpha = 0.05\), which means that gender is a significant variable to
whether an individual has had sexual intercourse.


\end{document}
